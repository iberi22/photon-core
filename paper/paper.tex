% arXiv Preprint Template for: Photon-Core
% Category: cs.ET (Emerging Technologies) or cs.IT (Information Theory)
%
% Compile with: pdflatex paper.tex && bibtex paper && pdflatex paper.tex && pdflatex paper.tex

\documentclass[11pt,a4paper]{article}

% ============================================================================
% PACKAGES
% ============================================================================
\usepackage[utf8]{inputenc}
\usepackage[T1]{fontenc}
\usepackage{amsmath,amssymb,amsfonts}
\usepackage{graphicx}
\usepackage{hyperref}
\usepackage{booktabs}
\usepackage{algorithm}
\usepackage{algpseudocode}
\usepackage{listings}
\usepackage{xcolor}
\usepackage{subcaption}
\usepackage[margin=1in]{geometry}

% Code listing style
\lstdefinestyle{rust}{
    language=C,
    basicstyle=\ttfamily\small,
    keywordstyle=\color{blue},
    commentstyle=\color{gray},
    stringstyle=\color{orange},
    numbers=left,
    numberstyle=\tiny,
    breaklines=true,
    frame=single
}

% ============================================================================
% METADATA
% ============================================================================
\title{Photon-Core: A Rust-based Simulation Framework for\\High-Density 5D Optical Data Encoding}

\author{
    Ivan Belalcazar\\
    \texttt{iberi22@gmail.com}\\
    Independent Researcher
}

\date{January 2026}

% ============================================================================
% DOCUMENT
% ============================================================================
\begin{document}

\maketitle

% ----------------------------------------------------------------------------
\begin{abstract}
We present \textbf{Photon-Core}, an open-source Rust framework for simulating 5-dimensional optical data storage. Our system encodes digital data into simulated photonic voxels using four physical properties: intensity, polarization angle, optical phase, and wavelength. Each voxel stores 8 bits (1 byte) through 2-bit quantization of each dimension. We demonstrate: (1) a bidirectional codec achieving $\sim$476 MB/s encoding throughput, (2) Reed-Solomon error correction for robustness against simulated readout noise, (3) inherent steganographic properties where data is unrecoverable without polarization information, and (4) bit error rate analysis under varying noise conditions. Our framework provides researchers with a fast, memory-efficient testbed for exploring encoding algorithms suitable for next-generation optical storage systems such as Project Silica and 5D glass memory.

\textbf{Keywords:} 5D optical storage, photonic voxel, data encoding, Rust, simulation, steganography
\end{abstract}

% ----------------------------------------------------------------------------
\section{Introduction}

Recent advances in femtosecond laser writing have enabled data storage in fused silica glass with unprecedented longevity and density \cite{wang2024}. Systems like Microsoft's Project Silica and academic demonstrations of "5D" glass memory exploit multiple physical dimensions—spatial position (X, Y, Z), polarization, and intensity—to achieve densities exceeding 360 TB per disc with lifetimes of billions of years.

However, the \textit{logical} encoding layer—how digital bits map to these physical properties—remains underexplored in open literature. Hardware papers focus on laser physics and material science, leaving a gap in accessible software tools for algorithm development.

\textbf{Contribution.} We address this gap with Photon-Core, a simulation framework that:
\begin{itemize}
    \item Defines a 16-byte aligned \texttt{PhotonicVoxel} data structure optimized for SIMD operations
    \item Implements 8-bit/voxel encoding using intensity, polarization, phase, and wavelength
    \item Provides configurable noise models and Reed-Solomon error correction
    \item Demonstrates steganographic data hiding through polarization dependency
\end{itemize}

% ----------------------------------------------------------------------------
\section{Related Work}

\textbf{Hardware Systems.} Wang et al. \cite{wang2024} demonstrated 5D storage using birefringent nanostructures. Microsoft's Project Silica \cite{silica2024} achieved 7 TB on a 75×75×2mm glass plate.

\textbf{Simulation Tools.} Existing optical simulations (COMSOL, Lumerical) focus on wave propagation physics, not data encoding logic. Our work complements these by providing a fast logical layer.

% ----------------------------------------------------------------------------
\section{System Design}

\subsection{The Photonic Voxel}

Our atomic storage unit is defined as:

\begin{lstlisting}[style=rust, caption={PhotonicVoxel struct definition}]
#[repr(C)]
pub struct PhotonicVoxel {
    pub intensity: f32,    // [0.25, 1.0] - 2 bits
    pub polarization: f32, // [0, PI) rad - 2 bits
    pub phase: f32,        // [0, 2*PI) rad - 2 bits
    pub wavelength: f32,   // nm - 2 bits
}
\end{lstlisting}

The 16-byte size ensures cache-line alignment for vectorized operations.

\subsection{Encoding Scheme}

Each byte maps to one voxel via bit-field extraction:

\begin{table}[h]
\centering
\begin{tabular}{lcc}
\toprule
\textbf{Property} & \textbf{Bits} & \textbf{Values} \\
\midrule
Intensity & 0-1 & 0.25, 0.50, 0.75, 1.00 \\
Polarization & 2-3 & 0°, 45°, 90°, 135° \\
Phase & 4-5 & 0, $\pi$/2, $\pi$, 3$\pi$/2 \\
Wavelength & 6-7 & 532nm, 650nm, 450nm, 800nm \\
\bottomrule
\end{tabular}
\caption{8-bit encoding map per voxel}
\end{table}

% ----------------------------------------------------------------------------
\section{Experimental Results}

\subsection{Performance}

Benchmarks on AMD Ryzen 9 / Intel i7 (single-threaded):

\begin{table}[h]
\centering
\begin{tabular}{lrr}
\toprule
\textbf{Operation} & \textbf{Time (1KB)} & \textbf{Throughput} \\
\midrule
Encode & 2.1 $\mu$s & 476 MB/s \\
Decode (noiseless) & 9.6 $\mu$s & 104 MB/s \\
Decode (with noise) & 22.5 $\mu$s & 44 MB/s \\
\bottomrule
\end{tabular}
\caption{Codec performance benchmarks}
\end{table}

\subsection{Bit Error Rate Analysis}

Figure~\ref{fig:ber} shows BER vs. noise amplitude. The codec tolerates up to 6\% noise before significant degradation.

% \begin{figure}[h]
% \centering
% \includegraphics[width=0.8\textwidth]{figures/ber_plot.png}
% \caption{Bit Error Rate vs. Noise Amplitude}
% \label{fig:ber}
% \end{figure}

\subsection{Steganography Verification}

When polarization is ignored during readout, data recovery fails completely—only 25\% of bits align by chance, rendering the output as noise.

% ----------------------------------------------------------------------------
\section{Conclusion}

Photon-Core provides an accessible, high-performance framework for 5D optical encoding research. Future work includes SIMD-accelerated batch encoding, GPU implementation, and integration with physical optical simulation tools.

\textbf{Code Availability:} \url{https://github.com/iberi22/photon-core}

% ----------------------------------------------------------------------------
\bibliographystyle{plain}
\begin{thebibliography}{9}

\bibitem{wang2024}
Y. Wang et al., "Ultrafast laser writing of 5D optical data storage in silica glass," \textit{Nature Photonics}, 2024.

\bibitem{silica2024}
Microsoft Research, "Project Silica: Storing data in glass," 2024. [Online]. Available: https://www.microsoft.com/en-us/research/project/project-silica/

\end{thebibliography}

\end{document}
